why am I implementing FBA analysis model into my dataset?
Because I need to have a better theoretical understanding.

Balances (conserved quantities) -> energy, mass, momentum, osmotic pressure, solvent capacity, electro-neutrality

Bounds (limit numerical ranges of variables) -> concentrations, fluxes, kinetic constants

constraints on the level of - diversity on production plans
							- resources
							intrinsic capabilities (how many links do we have in the network? Non-zero elements do we have in stochiometric matrix?)

19.03.21
An organism metabolic model of Homo Sapiens (738 metabolites, 1008 reactions) was downloaded from BIGG Models Database. Since its stoichiometric matrix density is quite small (0.00539), I have kept the size of the rows (738) but reduced the size of the columns from 1008 to 20 by randomly choosing non-zero elements from each row. My finalized stoic. matrix size is 738 x 20 (nodes/metabolites x features/flux exchanges).
Randomly created 10 objective functions are considered. Their coefficients vary from -2 to 2 and I kept the zero elements: non-zero elements ratio as 2:1 in objective functions.
I have assigned random boundaries for features.
With above-mentioned items, my solution vectors result in zero vectors. I am doing some minor adjustments however still could not obtain vectors with values yet.

20.03.21

I have used the complete stoichiometric matrix of Homo Sapiens organism metabolic model. After I had the solution vector results, defined the zero columns of solution vectors and erased them from the stoic. matrix and restarted the linear programming. In two steps, I have obtained a narrower solution vector (with less flux exchanges/features).