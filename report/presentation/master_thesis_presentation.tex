\documentclass[aspectratio=169]{beamer}
% Class options include: notes, notesonly, handout, trans,
%                        hidesubsections, shadesubsections,
%                        inrow, blue, red, grey, brown

% Other possible aspect ratio values are: 1610, 149, 54, 43 and 32. By default, it is to 128mm by 96mm(4:3).
\usefonttheme{professionalfonts}

% Theme for beamer presentation.
\usepackage{beamerthemesplit} 
% Other themes include: beamerthemebars, beamerthemelined, 
%                       beamerthemetree, beamerthemetreebars  
%\usepackage{amsmath}

\title{A Network-based Analysis of Technology-driven and Load-driven Constraints in Production Data}
\author{Serhat Kosif}
\institute{Jacobs University}
\date{Spring Semester 2021}

\begin{document}
	
	\begin{frame}[plain]
		\titlepage
	\end{frame}
	\note{Talk for 30 minutes} % Add notes to yourself that will be displayed when
	% typeset with the notes or notesonly class options
	
\section[Outline]{}
	
	% Creates table of contents slide incorporating
	% all \section and \subsection commands
	\begin{frame}
		\tableofcontents
	\end{frame}

\section[Introduction]{Introduction}
\subsection[Production Lines and Handling Types]{Production Lines and Handling Types}
\begin{frame}
	%\frametitle{Production Lines}   
	\begin{columns}[c]
		\column{.7\textwidth}  % slides are 3in high by 5in wide
		\hspace{-1.2cm}\includegraphics[width=11cm]{../images/steel-production-steps.png}
		\column{.3\textwidth}
		%\hspace{6cm}\includegraphics[height=3.7cm]{../tables/production_lines.png}
	\end{columns}
\end{frame}
\begin{frame}
	\begin{columns}[c]
		\column{.7\textwidth}  % slides are 3in high by 5in wide
		\hspace{-1.2cm}\includegraphics[width=11cm]{../tables/steel-production-steps_2.png}
		\column{.3\textwidth}
		%\hspace{6cm}\includegraphics[height=3.7cm]{../tables/production_lines.png}
	\end{columns}
\end{frame}
\begin{frame}
	\begin{columns}[c]
		\column{.7\textwidth}  % slides are 3in high by 5in wide
		\hspace{-1.2cm}\includegraphics[width=11cm]{../images/steel-production-steps_old.png}
		\column{.3\textwidth}
		\hspace{6cm}\includegraphics[height=3.7cm]{../tables/production_lines.png}
	\end{columns}
\end{frame}
\begin{frame}
	\begin{columns}[c]
		\column{.7\textwidth}  % slides are 3in high by 5in wide
		\hspace{-1.2cm}\includegraphics[width=11cm]{../tables/steel-production-steps_old_2.png}
		\column{.3\textwidth}
		\hspace{6cm}\centering\includegraphics[height=3.7cm]{../tables/production_lines.png}
	\end{columns}
\end{frame}
\begin{frame}
	\begin{columns}[c]
		\column{.7\textwidth}  % slides are 3in high by 5in wide
		\hspace{-1.2cm}\includegraphics[width=11cm]{../tables/steel-production-steps_old_2.png}
		\column{.3\textwidth}
		\hspace{6cm}\centering\includegraphics[height=3.7cm]{../tables/production_lines.png}
		\includegraphics[width=\linewidth]{../tables/production_lines_handling_types_2.png}
	\end{columns}
\end{frame}
%\note[enumerate]       % Add notes to yourself that will be displayed when
%{                      % typeset with the notes or notesonly class options
\section[Operations Research Model]{Operations Research Model}
\subsection[Steel Production Data Analysis]{Steel Production Data Analysis}
%\subsubsection[Association Networks]{Association Networks}
\begin{frame}
	\frametitle{Association Networks}
	\begin{columns}[c]
		\column{.5\textwidth}
		\centering
		\includegraphics[height=4.5cm]{../tables/arbitrary_production_data_set.png}
		\[Lift(A\leftrightarrow B)=\frac{P(A,B)}{P(A)*P(B)}\]
		\column{.6\textwidth}
	\end{columns}
\end{frame}
\begin{frame}
	\frametitle{Association Networks}
	\begin{columns}[c]
		\column{.5\textwidth}
		\centering
		\includegraphics[height=4.5cm]{../tables/arbitrary_production_data_set.png}
		\[Lift(A\leftrightarrow B)=\frac{P(A,B)}{P(A)*P(B)}\]
		\column{.6\textwidth}
		\includegraphics[width=8.5cm]{../tables/methodology-association-networks-adjacency_graph.png}
	\end{columns}
\end{frame}
%\subsubsection[Binning Schemes]{Binning Schemes}
\begin{frame}
	\frametitle{Binning Schemes}
	%\centering
	\includegraphics[width=\textwidth]{../tables/binning_schemes.png}
\end{frame}
%\subsubsection[Network Metrics Analysis]{Network Metrics Analysis}
\begin{frame}
	\frametitle{Network Metrics Analysis}
	\begin{columns}[c]
		\column{.55\textwidth}
		Modularity calculation was performed for the real network based on Newman (2006) formulated in his article.
		%\[Q = \frac {1} {4 m}\sum_ {ij} (A_{ij} - \frac {k_{i} k_{j}}{2 m}) s_{i} s_{j}\]
		\vspace{4cm}
		\column{.5\textwidth}
	\end{columns}
\end{frame}
\begin{frame}
	\frametitle{Network Metrics Analysis}
	\begin{columns}[c]
		\column{.55\textwidth}
		Modularity calculation was performed for the real network based on Newman (2006) formulated in his article.
		%\[Q = \frac {1} {4 m}\sum_ {ij} (A_{ij} - \frac {k_{i} k_{j}}{2 m}) s_{i} s_{j}\]
		\includegraphics[width=\linewidth]{../tables/cartoon-null-model-definitions.png}
		\column{.5\textwidth}
	\end{columns}
\end{frame}
\begin{frame}
	\frametitle{Network Metrics Analysis}
	\begin{columns}[c]
		\column{.55\textwidth}
		Modularity calculation was performed for the real network based on Newman (2006) formulated in his article.
		%\[Q = \frac {1} {4 m}\sum_ {ij} (A_{ij} - \frac {k_{i} k_{j}}{2 m}) s_{i} s_{j}\]
		\includegraphics[width=\linewidth]{../tables/cartoon-null-model-definitions.png}
		\column{.5\textwidth}
		\includegraphics[width=\textwidth]{../tables/expected_network_structures_1.png}
	\end{columns}
\end{frame}
\begin{frame}
	\frametitle{Network Metrics Analysis}
	\begin{columns}[c]
		\column{.55\textwidth}
		Modularity calculation was performed for the real network based on Newman (2006) formulated in his article.
		%\[Q = \frac {1} {4 m}\sum_ {ij} (A_{ij} - \frac {k_{i} k_{j}}{2 m}) s_{i} s_{j}\]
		\includegraphics[width=\linewidth]{../tables/cartoon-null-model-definitions.png}
		\column{.5\textwidth}
		\includegraphics[width=\textwidth]{../tables/expected_network_structures_2.png}
	\end{columns}
\end{frame}
\subsection[Abstract Theoretical Framework]{Abstract Theoretical Framework}
%\subsubsection[Genom-scale Cellular Networks]{Genom-scale Cellular Networks}
\begin{frame}
	%\centering
	\includegraphics[width=\textwidth]{../tables/gcn_intro.png}
\end{frame}
%\subsubsection[Flux Balance Analysis]{Flux Balance Analysis}
\begin{frame}
	%\centering
	\includegraphics[width=\textwidth]{../tables/fba_1.png}
\end{frame}
\begin{frame}
	%\centering
	\includegraphics[width=\textwidth]{../tables/fba_2.png}
\end{frame}
\begin{frame}
	%\centering
	\includegraphics[width=\textwidth]{../tables/fba_3.png}
\end{frame}
\begin{frame}
	%\centering
	\includegraphics[width=\textwidth]{../tables/fba_4.png}
\end{frame}
\begin{frame}
	%\centering
	\includegraphics[width=\textwidth]{../tables/fba_5.png}
\end{frame}
\begin{frame}
	%\centering
	\includegraphics[width=\textwidth]{../tables/fba_6.png}
\end{frame}
\subsection[Concepts Integration]{Concepts Integration}
\begin{frame}
	\centering
	\includegraphics[height=\textheight]{../images/methodology-ORmodel-cartoon_complete_framework.png}
\end{frame}
	
\section[Implementation, Analysis and Results]{Implementation, Analysis and Results}
	\subsection[Steel Production Events Analysis]{Steel Production Events Analysis}
	\begin{frame}
		\centering
		\includegraphics[width=\textwidth]{../tables/results-real_life_events_analysis-results.png}
	\end{frame}
	\subsection[Simulation Events Analysis]{Simulation Events Analysis}
	\begin{frame}
		\centering
		\includegraphics[width=\textwidth]{../tables/results-simulation-results.png}
	\end{frame}
	

\section[Conclusion and Outlook]{Conclusion and Outlook}
	\begin{frame}
		\begin{itemize}
			\item We linked these two data processing schemes: Fixed Step Sized binning and	Fixed Bucket Sized binning, to different constraint categories and obtained different results in the steel production events analysis.
			\item With Flux Balance Analysis, we constrained the system	at different levels and allowed for fluctuations in the input and controlled the products via objective functions.
			\item Further perturbation experiments can be performed with a randomly structured graph (set of production rules) instead of a ready model (as we used the homo sapiens metabolic model).
			\item The main challenge would be to construct arbitrary networks within the Operations Research framework. Therefore, different categories of constraints such as technical constraints, logistic constraints, physical and chemical constraints, economical constraints, and performance-indicator based constraints need to be quantified carefully in this framework.
		\end{itemize}
	\end{frame}

\begin{frame}{}
	\centering \Large
	\emph{Thank you for your attention.}
\end{frame}
	\note{The end}       % Add notes to yourself that will be displayed when
	% typeset with the notes or notesonly class options
	
\end{document}
