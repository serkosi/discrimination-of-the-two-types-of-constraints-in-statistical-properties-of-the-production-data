\chapter{Methodology}
{\color{red} 
	
	The initial step of this master thesis work was to quantify the characteristics of two hypothetical types of constraints in industrial production: technology-driven constraints and load-driven constraints. 
	
	I am planning to achieve this with an Operations Research Model consists of two steps. First, analyzing the statistical properties of association networks over Time in an extensive data set from steel manufacturing; second, developing an abstract theoretical framework to understand better the connection between each type of constraint and the statistical patterns created by them. 
	
}

{\color{red} 
	
	Introduce proposed concepts in the Operations Research Model (OR model).
	
	The OR model is a combination of Steel Manufacturing Events Analysis and Flux Balance Analysis. The art form of this model is to structure a standard data format and a shared analysis logic that allows comparing the results from manufacturing data and simulation data.
	
	A brief introduction for Association Networks and FBA. 
	An explanation for generating a data structure with OR-modeling in the combination of those. More detailed information to be given in the Backgroud Information Section, guiding the readers who have knowledge of FBA and Association Network concepts to the Concept Implementation Section.
	
	Usage of linear programming and generating sets of synthetic data allow comparing the statistical characteristics of their association network with those created from the real-world data set from steel manufacturing.
	
}

\section{Steel Production Events}
\subsubsection*{Association Networks}
\addcontentsline{toc}{subsubsection}{Association Networks}%
Beyond a simple network graph representation of historical production data, the formation of association networks is an insightful graph-based framework combining the tool: association rules and complex networks as Merten et al. (2020) performed in their article~\cite{MERTEN2020}. The relevant pipeline considers sequentially revealed events of a data set. It outputs a graph that unfolds the non-random occurrence of specific events together among the complete set that took place consecutively in the production period.

Assume we have a manufacturing data set with historical order, $D$, consists of $k$ sequences and $n$ events with mass values and sequence id's included as given in Table~\ref{Tab:D-dataset}.
\renewcommand{\arraystretch}{1.1}
\begin{table}[hb!]
	\centering
	\begin{tabular}{|cccccc|l}
		\cline{1-6}
		Event\_ID && Mass 	&& Sequence\_ID &  \\ \cline{1-6}
		1 	      && 280  	&& 1 		   	&  \\
		2 		  && 250	&& 1 		   	&  \\
		3 	      && 890	&& 2 		    &  \\
		4 		  && 850	&& 2 		    &  \\
		5 	      && 650	&& 2   		    &  \\
		6 	      && 745	&& 2 		    &  \\
		7 		  && 795	&& 2 		    &  \\
		8 		  && 150	&& 3 		    &  \\
		\vdots	  && \vdots && \vdots 	    &  \\
		n-4 	  && 940  	&& k-1	 	    &  \\
		n-3 	  && 540  	&& k			&  \\
		n-2 	  && 520	&& k 		    &  \\
		n-1       && 630	&& k 		    &  \\
		n 		  && 610	&& k 		    &  \\ \cline{1-6}
	\end{tabular}
	\caption{Arbitrarily Created Data Set $D$.}
	\label{Tab:D-dataset}
\end{table}

Examining the data set, one can say that the events with mass values: $890$, $850$, $650$, $745$ or $540$, $520$, $630$, $610$ are close to each other; thus, they are produced together and likely occur in the identical sequences among the complete data. In a further step, one can label the events mentioned above with a value interval (the so-called binning size) typical for every mass value with a tiny difference to each other. Binning generation for the events allows us to investigate them in a mass-production manner. Alternative binning methods will be addressed in the following subsection. 

One can hypothetically argue that the information mentioned above patterns are probably deliberate planning choices based on the related constraints acting on the manufacturing process performance. However, forming prevailing arguments is not a simple task for large and complicated data sets. Such a real-life data set may consist of more than $300,000$ events likely to have various events aggregated randomly in its large sequence groups.

To distinguish random co-occurrences from meaningful ones in production sequences and assess the complexity of production patterns before creating the network graphs, we extract the association rule from the set of sequences. With a similar approach as Merten et al. (2020) applied~\cite{MERTEN2020}, an association rule measure known as "Lift" was picked and calculated for every possible pairwise subset of events that occurred in the same sequences. By having a natural threshold of Lift measure $1$. The lift can be computed by the ratio of pair items joint probability divided by the multiplication of each item's marginal probability as
\begin{equation} \tag{6}
	Lift(A\leftrightarrow B)=\frac{P(A,B)}{P(A)*P(B)},
	\label{lift}
\end{equation}
thus, in the case of $Lift(A\leftrightarrow B)> 1$, B occurs likely if A occurs whereas $Lift(A\leftrightarrow B)< 1$, B unlikely occurs if A occurs. Indication of random and non-random co-occurrences as $0$ and $1$ in an adjacency matrix will provide the data structure to form a network as shown in Fig.~\ref{figure-adjacency_graph}.

%\begin{table}[]
%	\begin{tabular}{|c|cccccc|}
%		\hline
%		Events & 1   & 2   & 3   & \dots & n-1 & n   \\ \hline
%		1      & 0   & 1   & 0   & \dots & 0   & 0   \\
%		2      & 1   & 0   & 0   & \dots & 0   & 0   \\
%		3      & 0   & 0   & 0   & \dots & 0   & 0   \\
%		\vdots & \vdots & \vdots & \vdots & \ddots & \vdots & \vdots \\
%		n-1    & 0   & 0   & 0   & \dots & 0   & 1   \\
%		n      & 0   & 0   & 0   & \dots & 1   & 0   \\ \hline
%	\end{tabular}
%\end{table}
 \begin{figure}[!ht]
	\begin{center}
		\makebox[\textwidth]{
			\centering
			\includegraphics[width=0.7\linewidth]{../images/methodology-association-networks-adjacency_graph.png}}
		\caption{An Arbitrary Representation for Adjacency Matrix and Its Graph.}
		\label{figure-adjacency_graph}
	\end{center}
\end{figure}
\subsection{Binning Schemes}
%\addcontentsline{toc}{subsection}{Binning Methods}%

The data set $D$ events can be labelled with a typical value interval (the so-called binning size) for the Feature-A values with a slight difference. Binning generation can be performed in alternative ways, allowing us to put the hypothetically created constraints into practice.

Say that we do the Feature-A values labelling with a typical binning size, in our case, 99, so that all of the events in $D$ must match the corresponding \ac{fss} interval, as shown in Table~\ref{Tab: D-dataset-FSS}. 
\begin{table}[ht!]
	\centering
	\setlength{\arrayrulewidth}{0.79pt}%
	\caption{Data Set D with FSS Bin Size Labels.} 
	\begin{tabular}{|cc|c|ccc|c|}
		\hline \rowcolor[HTML]{FFFFC7}
		\makecell{Event\\ID} 	&& Feature-A    	&& FSS Bins && \makecell{Sequence\\ID}  \\ \hline
		1 	      && 280	    && 200-299	&& 1 		     \\
		2 		  && 250	    && 200-299	&& 1 		     \\
		3 	      && 890	    && 800-899	&& 2 		     \\
		4 		  && 850	    && 800-899	&& 2 		     \\
		\vdots	  && \vdots  	&& \vdots	&& \vdots 	     \\
		n-2 	  && 520	    && 500-599	&& k 		     \\
		n-1       && 630	    && 600-699	&& k 		     \\
		n 		  && 610	    && 600-699	&& k 		     \\ \hline
	\end{tabular}
	\label{Tab: D-dataset-FSS}
\end{table}

An alternative way of label generation is to create bins with equal event counts per bin among the complete data set; \ac{fbs} labelling is shown in Table~\ref{Tab: D-dataset-FBS}.
\begin{table}[ht!]
	\centering
	\setlength{\arrayrulewidth}{0.75pt}%
	\caption{Data Set D with FBS Bin Size Labels.}
	\begin{tabular}{|cc|c|ccc|c|}
		\hline \rowcolor[HTML]{FFFFC7}
		\makecell{Event\\ID} 	&& Feature-A    	&& FBS Bins && \makecell{Sequence\\ID} \\ \hline
		1 	      && 280	    && 200-599	&& 1 		     \\
		2 		  && 250	    && 200-599	&& 1 		     \\
		3 	      && 890	    && 630-899	&& 2 		     \\
		4 		  && 850	    && 630-899	&& 2 		     \\
		\vdots	  && \vdots  	&& \vdots	&& \vdots 	     \\
		n-2 	  && 520	    && 200-599	&& k 		     \\
		n-1       && 630	    && 630-899	&& k 		     \\
		n 		  && 610	    && 600-629	&& k 		     \\ \hline
	\end{tabular}
	\label{Tab: D-dataset-FBS}
\end{table}
The alternative binning generation methods mentioned above let us derive two distinguished approaches to construct association networks. The first one is the \acs{fss} Network approach; it has graph nodes as binning groups with equal bin sizes. Manipulation of binning size allows us to aggregate events in different network nodes. The \acs{fbs} Network approach is the second one where the network nodes are binning groups with an equal number of events per bin. Defining a typical bucket size for the network nodes results in arbitrary interval boundaries for each node, and it allows to control their population.
 \begin{figure}[!ht]
	\begin{center}
		\makebox[\textwidth]{
			\centering
			\includegraphics[width=0.8\linewidth]{../images/methodology-association-networks-hyp_networks.png}}
		\caption{Graph Results For Two Different Network Approaches.}
		\label{figure-hyp_graphs}
	\end{center}
\end{figure}

Constructing \acs{fss} and \acs{fbs} networks concerning technology-driven constraints and load-driven constraints, respectively, for the production events, underlie the developed hypothesis of this thesis work: Non-random features of the association networks derived from these two approaches.
\subsection{Network Metrics Analysis}
%\addcontentsline{toc}{subsection}{Network Metrics Analysis}%
As explained in the previous subsection, one can label a data set differently and generate identical graphs with \acs{fss} Network and \acs{fbs} Network approaches. We argue that resultant graphs have various motifs which are non-trivial and emerge from the statistical patterns in data. In further steps in this subsection, we review a well-known network measure: modularity and statistical techniques of randomness control to be integrated into our analysis pipeline.
\subsubsection*{Modularity Measure}
The variety of textures arises from how the nodes are clustered within their neighbourhood or having different degree values. The degree is a network metric that quantifies one node's links (or edges) to the other nodes~\cite{Barabasi2016}. The degree distribution of the network gives an idea about the connectivity patterns within the network. It allows us to distinguish the nodes with a high degree from the nodes with a low degree.

Identification of tightly connected node groups is a way of quantifying community structure in networks~\cite{Girvan7821}. Communities (the so-called modules) are groups of nodes that probably play similar roles within the graph~\cite{FORTUNATO201075}. Modularity is a network measure for community detection and quantifies the strength of community structure in that specific network. It is a way to express the network characteristics.

Newman (2006) formulated modularity in his article as
\begin{equation} %\tag{2}
	Q = \frac {1} {4 m}\sum_ {ij} (A_{ij} - \frac {k_{i} k_{j}}{2 m}) s_{i} s_{j} ,
	\label{modularity}
\end{equation}
\myequations{Modularity Formula} 
where the network graph has an $m$ number of edges, and $A_{ij}$ is the number of edges between vertices $i$ and $j$. $A_{ij}$ is the element of the adjacency matrix introduced in Fig.~\ref{figure-adjacency_graph}. It can be $0$ or $1$. $k_{i}$ and $k_{j}$ are the vertex degrees, and ${k_{i} k_{j}}/{2 m}$ is the expected number of edges between $i$ and $j$ if edges are randomly placed. $s_{i}$ and $s_{j}$ are the divided network groups. They are equal to $1$ if $i$ and $j$ belong to the same group and $0$ otherwise. Eq.\eqref{modularity} is used to separate the network into two communities only; however, many networks may contain more than two communities. Therefore, a repeated division into two is adapted: dividing the network into two graphs, then the two sub-graphs further divided into two only if that would maximise $Q$. After first partitioning, the edges falling between the further divided sub-graphs are neglected, leading to a wrong maximisation quantity. For this reason, the author introduced the additional contribution $\Delta Q$.~\cite{Newman8577}
% $B_{ij} = A_{ij} - \frac {k_ {i} k_ {j}} {2 m}$\\
% $Q = \frac {1} {4 m} s^{T} Bs = \frac {1} {4 m}\sum_ {i = 
%	1}^{n} (u_ {i}^{T} . s)^{2}\beta_ {i}$\\
% $\Delta Q = \frac {1} {4 m} s^{T} B^{(g)} s$\\
% $B_{ij}^{(g)} = B_{ij} - \delta_{ij}\sum_ {k\in g} B_{ik}$

The formulation given in Eq.\eqref{modularity} was used in this thesis work to calculate the modularity of the association networks. Since the results obtained with the combination of $Q$ and $\Delta Q$ do not significantly differ from the results obtained only using $Q$, the modularity calculations in this work were performed with the latter one to lower the computation timing.

\subsubsection*{Randomness Control Concerning Different Null Models}
As Eq.~\eqref{modularity} gives a clue, one can measure a real network modularity quality by comparing it with the community structure in a random graph~\cite{GirvanNewman2004}. The distribution of degrees in random graphs is highly homogeneous, and they do not reveal a significant level of order or organisation~\cite{FORTUNATO201075}. Various sophisticated random graphs (the so-called null models) can be generated from the original network graph by keeping some of its structural properties the same~\cite{Maslov910, MERTEN2020, FORTUNATO201075, Enders2018}.
 \begin{figure}[!ht]
	\begin{center}
		\makebox[\textwidth]{
			\centering
			\includegraphics[width=0.9\linewidth]{../images/cartoon-null-model-definitions.png}}
		\caption{Formation of Different Null Models.}
		\label{figure-null_models}
	\end{center}
\end{figure}

Our analysis pipeline considers two types of randomised graphs for our association networks: \ac{nmd} and \ac{nmm}, as shown in Fig.~\ref{figure-null_models}. In \acs{nmd}, all edges that belong to the real network are shuffled in a pairwise fashion by keeping the original degrees sequence which allows conserving any possible skewed degree distribution in the real network~\cite{Maslov910, Fretter2012, FORTUNATO201075}. In \acs{nmm}, intra-edges and inter-edges among modules are shuffled separately by preserving the original degrees sequence~\cite{Fretter2012}. We should emphasise an essential detail in our design decision that might affect the results; \acs{nmm} keeps inter-edges from different module pairs together while the shuffling process, even if there are more than two modules in the real network. However, some module pairs might be strongly interconnected in most realistic situations, while the others are almost not linked to each other.
 \begin{figure}[!ht]
	\begin{center}
		\makebox[\textwidth]{
			\centering
			\includegraphics[width=0.7\linewidth]{../images/methodology-network-metrics-analysis-normal_distribution.png}}
		\caption{Chart Comparing the Various Grading Methods in A Normal Distribution.}
		\label{figure-normal_distribution}
	\end{center}
\end{figure}


One thousand random graphs concerning the respective null model constraints are created, and their modularity values are computed to compare with the real network. The histogram of resulted modularity values converges to a normal distribution like the one shown in Fig.~\ref{figure-normal_distribution}. One can quantify the real network randomness in the context of the respective null model by computing the \acl{zscore} (the so-called \acs{zscore}), $z$, as
\begin{equation} %\tag{2}
	z = \frac{Q-\mu}{\sigma}.
	\label{zscore}
\end{equation} 
\myequations{Standard Score Formula} 

$Q$ is the modularity value for the real network; $\mu$ is the expectation value (mean) of $Q$ in the set of $1000$ randomised graphs. $\sigma$ is the standard deviation of $Q$ in the randomised graphs.

$z$ is the number of standard deviations by which the real network modularity value is below or above the expected value, $\mu$. The \acs{zscore} lower than $1$ or higher than $-1$ indicate that the real network is incidental; the \acs{zscore} between $1$ and $2$ or between $-2$ and $-1$ suggest that the real network is close to random characteristics. In contrast, the \acs{zscore} greater than $2$ or less than $-2$ indicate a significant deviation from randomness.

\acs{nmd} is the null model that gives information about the modularity since it destroys the modules in the real network while randomising it. \acs{nmm} is essentially the control null model to detect strange effects and whether it is meaningful to discuss modularity. Comparing a real modular network with \acs{nmm} random graphs will lead the \acs{zscore} to zero, no matter the actual modularity value. If the \acs{zscore} using \acs{nmm} is drastically away from zero, then the type of modularity in the real network graphs are somewhat different and very complicated.

In some networks, small groups of nodes organised by following a hierarchical rule~\cite{BARABASI2001559} form large groups displaying a high degree of clustering while the degree distribution follows a power law~\cite{Barabasi2003}. That hierarchical organisation of nodes creates a nested modularity structure in the networks, having modules within modules. That type of organisation is observed in several real networks like the World wide web, the Internet at the domain level, actor-network~\cite{Barabasi2003}, macaque \& cat cortical systems~\cite{Young2000} and the Escherichia coli metabolic network~\cite{Ravasz1551}. We assume that the real network has a complicated structure or hierarchical organisation since the randomising scheme would destroy the internal modularity of graph modules if the \acs{zscore} concerning \acs{nmm} is lower than $-2$ or higher than $2$. 

Table~\ref{Tab: z-score_range_def} summarises possible network structures, including our assumptions for the respective \acs{zscore} intervals under the effect of null model choice.

\begin{table}[ht!]
	\centering
	\setlength{\arrayrulewidth}{0.79pt}%
	\caption{Expected Network Structures with Respect to Null Models.}
	\begin{tabular}{|
			>{\columncolor[HTML]{FFFFC7}}c |c|c|}
		\hline
		\begin{tabular}[c]{@{}c@{}}Real \\ Network   \\ Z-score\end{tabular} & \cellcolor[HTML]{FFFFC7}\begin{tabular}[c]{@{}c@{}}Conserving \\ Degrees Sequence \\ (NM-d)\end{tabular} & \cellcolor[HTML]{FFFFC7}\begin{tabular}[c]{@{}c@{}}Conserving \\ Degrees Sequence \\ and Modules (NM-m)\end{tabular} \\ \hline
		$z\leq-2$ & \begin{tabular}[c]{@{}c@{}}nonrandom,\\ non-modular,\\ hierarchical\end{tabular} & \begin{tabular}[c]{@{}c@{}}nonrandom,\\ non-modular,\\ hierarchical\end{tabular} \\ \hline
		$-2<z<2$ & \begin{tabular}[c]{@{}c@{}}random,\\ non-modular\end{tabular} & \begin{tabular}[c]{@{}c@{}}simple,\\ non-hierarchical\end{tabular} \\ \hline
		$z\geq2$ & \begin{tabular}[c]{@{}c@{}}nonrandom,\\ modular,\\ simple\end{tabular} & \begin{tabular}[c]{@{}c@{}}nonrandom, \\modular,\\ hierarchical\end{tabular} \\ \hline
	\end{tabular}
	\label{Tab: z-score_range_def}
\end{table}

\subsection{Flux Balance Analysis}
The genome-scale integrated networks are necessary tools used by metabolic engineers on model generation, theoretical and computational analysis for microbial organisms. In addition, the network theory tools expand the feasible space for the following analysis techniques in the field. 

\textcolor{red}{Introducing stoic. matrix. Explain how the two graphs are formally obtained by manipulated the stoichiometric matrix, explain the metabolite-centric network is $S*S^{T}$ and binarized. In contrast, the reaction-centric network is $S^{T}*S$ and binarized. Introduce the general idea of FBA as an optimization scheme in a steady-state solution space.}
Although the networks shown in Fig.~\ref{figure-metabolic-networks} do not contain any information about directionality or effectiveness of the reactions to the system, the set of rules take place in networks can be represented in more detail and stoichiometrically by an m-by-r matrix formulation (the so-called Stoichiometric Matrix $S$), whereas its column elements represent reactions that play a role in the chemical transformation, and its row elements represent metabolites as
\begin{equation} \tag{1}
	S =  \begin{bmatrix} 
		s_{11} & s_{12} & \dots  & s_{1r}\\
		s_{21} & s_{22} & \dots  & s_{2r}\\
		\vdots & \vdots &\ddots & \vdots \\
		s_{m1} & s_{m2} & \dots & s_{mr} 
	\end{bmatrix}=(s_{ij})\in \mathbb{Z}^{mxr},
	\label{stoichio}
\end{equation}

Fig.~\ref{figure-metabolic-networks} shows two differently constructed networks showing interactions between metabolites, intermediate or end products and metabolic reactions for a particular metabolism: Homo Sapien. In Fig.~\ref{figure-metabolic-centric} the graph nodes stand for the metabolites, graph edges are the reactions. In contrast, in Fig.~\ref{figure-reaction-centric} the roles are reversed so that the graph edges represent the metabolites, and the graph nodes represent the reactions.

\begin{figure}[!ht]
	\centering
	\begin{subfigure}{0.5\textwidth}
		\includegraphics[width=1\linewidth]{../images/methodology-ORmodel-metabolic_centric_network.png}
		\caption{Metabolic-centric Network}
		\label{figure-metabolic-centric}
	\end{subfigure}\hfill% or \hspace{5mm} or  \hspace{0.3\textwidth}
	\begin{subfigure}{0.5\textwidth}
		\includegraphics[width=1\linewidth]{../images/methodology-ORmodel-reaction_centric_network.png}
		\caption{Reaction-centric Network}
		\label{figure-reaction-centric}
	\end{subfigure}
	\caption{Network Representations for Homo Sapiens Metabolic Model}
	\label{figure-metabolic-networks}
\end{figure}

Studying biological metabolic systems, generated models to achieve cellular objectives like cell growth or ATP production brings the necessity of various tools to analyze reconstructed genome-scale networks.~\cite{KIM, HAO}. One of the commonly used tools is Flux Balance Analysis (FBA). It is a constraint-based modelling approach to simulate microbial metabolisms and can be applied to biochemical-reaction networks containing the chemical transformations and flux exchanges in that particular network~\cite{KAUFFMAN2003491, PRICE2004}.

while one can express the fluxes in a one-dimensional array (the so-called Flux Vector $V$) as 
\begin{equation} \tag{2}
	V = \begin{bmatrix}
		v_{1} \\
		v_{2} \\
		\vdots \\
		v_{r}
	\end{bmatrix}=(v_{i})\in \mathbb{R}.
	\label{solutionvector}
\end{equation}
$V$ contains flux exchange values for the corresponding reactions in the system and gives information about the flux distribution; hence, those can be both positive and negative real numbers. Definition of a mass balance ($S.V=0$) constraint in the FBA enables us to analyze the metabolic network operations in a steady-state~\cite{KAUFFMAN2003491,PRICE2004}.
%(resulting steady state vectors/resulting optimized solution vectors)
\begin{equation} \tag{3}
	S.V = \begin{bmatrix} 
		s_{11}v_{1} + s_{12}v_{2} + \dots + s_{1r}v_{r} \\
		s_{21}v_{1} + s_{22}v_{2} + \dots + s_{2r}v_{r} \\
		\vdots \\
		s_{m1}v_{1} + s_{m2}v_{2} + \dots + s_{mr}v_{r} 
	\end{bmatrix}=
	\begin{bmatrix} 
		0 \\
		0 \\
		\vdots \\
		0
	\end{bmatrix}.
	\label{massbalanceconstraint}
\end{equation}
The higher amount of metabolite consideration in the set of rules, $S$, in other words, the larger matrix size by its rows amount means the more complex organization structure taken into account while preserving the steady-state in the whole system.

More than one steady-state solution might be present since it is impossible to identify all constraints in a cellular system~\cite{KAUFFMAN2003491}. Therefore, one can formulate an optimization approach to identify reaction network steady-states that maximize the biomass~\cite{KAUFFMAN2003491,PRICE2004} or control the production of specific metabolites~\cite{VARMA1993} within a defined objective function under the consideration of the system constraints. According to Price et al. (2004),
there are three primary purposes to generate objective functions: to discover allowable characteristic properties in the genome-scale network reconstruction; to mimic probable physiological functions like biomass or ATP production to be able to determine likely physiological states; and lastly, to design a genetic variant or sub-type to obtain a desired particular product~\cite{PRICE2004}.

The objective function can be thought as a production plan that gives an idea about the diversity of products that the relevant system can produce, and one can express its coefficients in a one-dimensional array as
\begin{equation} \tag{4}
	O =  \begin{bmatrix}
		o_{1} & o_{2} & \dots  & o_{r}\\
	\end{bmatrix}=(o_{i})\in \mathbb{R}.
	\label{objectivecoefficients}
\end{equation}
As given in Eq.\eqref{biomassmaximisation}, the Objective Function, $Z$, rules the maximized output based on its non-zero coefficients, which are the decisive ones for the flux elements of $V$ to be considered.
\begin{equation} \tag{5}
	Z = O.V = (o_{1}v_{1} + o_{2}v_{2} + \dots + o_{r}v_{r})\in \mathbb{R}_{\ge0}.
	\label{biomassmaximisation}
\end{equation}
Stoichiometry and mass-balance are the constraints introduced so far in Eq.\eqref{stoichio} and Eq.\eqref{massbalanceconstraint}. In addition, upper and lower bounds are introduced for particular fluxes in $V$ during the optimization process. The bounds are used in the reactions for uptake and secretion of any organic metabolite. In the uptake reactions, nutrients are transported to the inside of the metabolic network. In the secretion reactions, products are exported to the outside of the network. The rest of the fluxes in $V$ are used in the exchange reactions, namely the intermediate reactions in the network. The constraints are decisive on the reactions for uptake and secretion, whereas no limitation is considered in the exchange reactions. Quantification of imported nutrients and exported outputs (the so-called Resources and Wastes) by constraining them with upper and lower bounds to fulfil a single objective function goal might play a significant role in the optimization process.

 \begin{figure}[!ht]
	\begin{center}
		\makebox[\textwidth]{
			\centering
			\includegraphics[width=0.7\linewidth]{../images/methodology-ORmodel-uptake_secretion_cartoon.png}}
		\caption{A Simplified Reaction-centric Network Sketch Shows the Reactions for Exchange, Uptake and Secretion.}
		\label{figure-uptake-secretion-cartoon}
	\end{center}
\end{figure}

The above-explained optimization process is a linear programming problem since the mass balance (Eq.\eqref{massbalanceconstraint}), the Objective Function (Eq.\eqref{biomassmaximisation}), and linear equations formulate the upper \& lower bounds for fluxes. The linear optimization result maximizes the structured Objective Function in the form of a flux distribution~\cite{KAUFFMAN2003491,PRICE2004}. Since each term in Eq.\eqref{biomassmaximisation} is a produced biomass expression for the fluxes, the summation of those terms will give the overall growth of the system for a single network state.

\subsubsection*{Fluxes for Uptake and Secrete Reactions}
\addcontentsline{toc}{subsubsection}{Fluxes for Uptake and Secrete Reactions}%
Let
\begin{equation} \tag{7}
	V^{*}=(v^{*}_{1}, v^{*}_{2},\dots, v^{*}_{x})= (a_{i}\le v^{*}_{i}\le b_{i})\in V
	\label{constrainedfluxlist}
\end{equation}
is a set of fluxes picked from $V$ to be limited with the bounds: $a_{i}$ and $b_{i}$ which are used in the reactions for uptake and secretion as previously introduced.
\subsubsection*{Upper and Lower Bounds}
\addcontentsline{toc}{subsubsection}{Upper and Lower Bounds}%
\subsubsection*{Objective Functions}
\addcontentsline{toc}{subsubsection}{Objective Functions}%
