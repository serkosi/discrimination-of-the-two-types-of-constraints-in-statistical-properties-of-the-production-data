\section{Integration of Concepts}
As explained in the previous section, we obtain maximised flux distribution of the metabolic organism for a single network state by performing linear optimisation. This section clarifies how the outputs from linear optimisation in more than one run are converted into a compatible format to construct data structures similar to real-life events. With this integration step, one can construct association networks derived from \acs{fss} and \acs{fbs} labels of the generated data and calculate modularity values concerning alternative null models.

 \begin{figure}[!ht]
	\begin{center}
		\makebox[\textwidth]{
			\centering
			\includegraphics[width=1\linewidth]{../images/methodology-ORmodel-cartoon_complete_framework.png}}
		\caption{Simulation Model Illustration}
		\label{figure-complete_framework_cartoon}
	\end{center}
\end{figure}

%\begin{table}[hb!]
%	\centering
%	\setlength{\arrayrulewidth}{0.5pt}% 
%	\begin{tabular}{|ccc|}
%		\hline \rowcolor[HTML]{FFFFC7}
%		Biomass & FSS Bins	& \makecell{Seq.\\ID}  	\\ \hline
%		$(O.V)_{1}$	    & $s\_label_{1}$	& 1 		\\
%		$(O.V)_{2}$	    & $s\_label_{2}$	& 1 		\\
%		$(O.V)_{3}$	    & $s\_label_{3}$	& 2 		\\
%		\vdots  		& \vdots		& \vdots 	\\
%		$(O.V)_{10000}$	& $s\_label_{x}$	& 200 		\\ \hline
%	\end{tabular}
%\end{table}
%\begin{table}[hb!]
%	\centering
%	\setlength{\arrayrulewidth}{0.5pt}% 
%	\begin{tabular}{|ccc|}
%		\hline \rowcolor[HTML]{FFFFC7}
%		Biomass & FBS Bins	& \makecell{Seq.\\ID}  	\\ \hline
%		$(O.V)_{1}$	    & $b\_label_{1}$	& 1 		\\
%		$(O.V)_{2}$	    & $b\_label_{2}$	& 1 		\\
%		$(O.V)_{3}$	    & $b\_label_{3}$	& 2 		\\
%		\vdots  		& \vdots		& \vdots 	\\
%		$(O.V)_{10000}$	& $b\_label_{y}$	& 200 		\\ \hline
%	\end{tabular}
%\end{table}

%\begin{equation} %\tag{8}
%	(O.V)_{1}= (o_{1,1}v_{1} + o_{1,2}v_{2} + \dots + o_{1,r}v_{r})
%\end{equation}

%\begin{equation} %\tag{8}
%	(O.V)_{2}= (o_{2,1}v_{1} + o_{2,2}v_{2} + \dots + o_{2,r}v_{r})
%\end{equation}

%\begin{equation} 	
%	(O.V)_{50}= (o_{50,1}v_{1} + o_{50,2}v_{2} + \dots + o_{50,r}v_{r})
%\end{equation}

%\begin{equation} 	
%	(O.V)_{51}= (o_{1,1}v_{1} + o_{1,2}v_{2} + \dots + o_{1,r}v_{r})
%\end{equation}

%\begin{equation} 	
%	(O.V)_{100}= (o_{50,1}v_{1} + o_{50,2}v_{2} + \dots + o_{50,r}v_{r})
%\end{equation}

%\begin{equation} 	
%	(O.V)_{10000}= (o_{50,1}v_{1} + o_{50,2}v_{2} + \dots + o_{50,r}v_{r})
%\end{equation}

Fig.~\ref{figure-complete_framework_cartoon} illustrates the complete simulation model. \acs{fba} optimisation scheme is summarised in Fig.~\ref{figure-complete_framework_cartoon}a with a defined set of rules and system constraints, as introduced in the previous section. The overall growth of the biomass (Eq.~\eqref{biomassmaximisation}) is obtained with a single run of the optimisation algorithm. The resultant value is a simulation event conceptually equivalent to a steel production event. The optimisation algorithm is run $10000$ times to create a data set with $10000$ events. 

Fig.~\ref{figure-complete_framework_cartoon}b shows the data structure generation by introducing a production sequence concept. In steel manufacturing, each production sequence shows consistency among its events based on the production constraints as introduced in Section~\ref{background_information}; therefore, the random choice for non-zero coefficients in the objective function is kept fixed only for the events in the same sequence. Hence, the optimisation scheme can use various flux series to be considered in the biomass for created events in different sequences. 

Fig.~\ref{figure-complete_framework_cartoon}c shows the generated data labelled in alternative ways, as introduced in Tables~\ref{Tab: D-dataset-FSS} and \ref{Tab: D-dataset-FBS}, which is convenient to construct graphs to be analysed.