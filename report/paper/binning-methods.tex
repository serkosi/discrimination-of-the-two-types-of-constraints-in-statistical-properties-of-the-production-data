\subsection*{Binning Methods}
\addcontentsline{toc}{subsection}{Binning Methods}%

The data set $D$ events can be labelled with a typical value interval (the so-called binning size) for every Feature-A value with a slight difference to each other. Binning generation for the events allows us to investigate them in a sequential manufacturing system and can be performed in alternative ways.

Say that we do the Feature-A values labelling with a typical binning size, in our case, 99, so that all of the events in $D$ must match the corresponding step interval, Fixed Step Size (FSS), as shown in Table~\ref{Tab: D-dataset-FSS}. 
\begin{table}[hb!]
	\centering
	%	\caption*{$D$ Labeled with FSS Bin Sizes}
	\begin{tabular}{|ccccccc|l}
		\cline{1-7}
		\makecell{Event\\ID} 	&& Feature-A    	&& \makecell{FSS\\Bin Size}&& \makecell{Sequence\\ID} &  \\ \cline{1-7}
		1 	      && 280	    && 200-299	&& 1 		   &  \\
		2 		  && 250	    && 200-299	&& 1 		   &  \\
		3 	      && 890	    && 800-899	&& 2 		   &  \\
		4 		  && 850	    && 800-899	&& 2 		   &  \\
		\vdots	  && \vdots  	&& \vdots	&& \vdots 	   &  \\
		n-2 	  && 520	    && 500-599	&& k 		   &  \\
		n-1       && 630	    && 600-699	&& k 		   &  \\
		n 		  && 610	    && 600-699	&& k 		   &  \\ \cline{1-7}
	\end{tabular}
	\caption{Data Set D with FSS Bin Size Labels.}
	\label{Tab: D-dataset-FSS}
\end{table}

An alternative way of label generation is to create bins with equal event counts per bin among the complete data set, Fixed Bucket Size (FBS) given in Table~\ref{Tab: D-dataset-FBS}.
\begin{table}[ht!]
	\centering
	%	\caption*{$D$ Labeled with FBS Bin Sizes}
	\begin{tabular}{|ccccccc|l}
		\cline{1-7}
		\makecell{Event\\ID} 	&& Feature-A    	&& \makecell{FBS\\Bin Size}&& \makecell{Sequence\\ID} &  \\ \cline{1-7}
		1 	      && 280	    && 200-599	&& 1 		   &  \\
		2 		  && 250	    && 200-599	&& 1 		   &  \\
		3 	      && 890	    && 630-899	&& 2 		   &  \\
		4 		  && 850	    && 630-899	&& 2 		   &  \\
		\vdots	  && \vdots  	&& \vdots	&& \vdots 	   &  \\
		n-2 	  && 520	    && 200-599	&& k 		   &  \\
		n-1       && 630	    && 630-899	&& k 		   &  \\
		n 		  && 610	    && 600-629	&& k 		   &  \\ \cline{1-7}
	\end{tabular}
	\caption{Data Set D with FBS Bin Size Labels.}
	\label{Tab: D-dataset-FBS}
\end{table}
The alternative binning generation methods mentioned above let us derive two distinguished network approaches. The first one is the FSS Network; it has graph nodes as binning groups with equal bin sizes. Manipulation of binning size allows us to aggregate events in different network nodes. The second one is the FBS Network; its nodes are binning groups with an equal number of events per bin. Defining a typical bucket size for the network nodes results in arbitrary interval boundaries for each node, and it allows to control their population.
 \begin{figure}[!ht]
	\begin{center}
		\makebox[\textwidth]{
			\centering
			\includegraphics[width=0.8\linewidth]{../images/methodology-association-networks-hyp_networks.png}}
		\caption{Graph Results For Two Different Network Approaches.}
		\label{figure-hyp_graphs}
	\end{center}
\end{figure}

{\color{red}FSS and FBS networks generation for the production events underlie the developed hypothesis of this thesis work: Non-random features of the association networks derived from these two methods.}