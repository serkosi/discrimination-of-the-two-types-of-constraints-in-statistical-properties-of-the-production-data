\subsection{Binning Schemes}
%\addcontentsline{toc}{subsection}{Binning Methods}%

The data set $D$ events can be labelled with a typical value interval (the so-called binning size) for the Feature-A values with a slight difference. Binning generation can be performed in alternative ways, allowing us to put the hypothetically created constraints into practice.

Say that we do the Feature-A values labelling with a typical binning size, in our case, 99, so that all of the events in $D$ must match the corresponding \ac{fss} interval, as shown in Table~\ref{Tab: D-dataset-FSS}. 
\begin{table}[ht!]
	\centering
	\setlength{\arrayrulewidth}{0.79pt}%
	\caption{Data Set D with FSS Bin Size Labels.} 
	\begin{tabular}{|cc|c|ccc|c|}
		\hline \rowcolor[HTML]{FFFFC7}
		\makecell{Event\\ID} 	&& Feature-A    	&& FSS Bins && \makecell{Sequence\\ID}  \\ \hline
		1 	      && 280	    && 200-299	&& 1 		     \\
		2 		  && 250	    && 200-299	&& 1 		     \\
		3 	      && 890	    && 800-899	&& 2 		     \\
		4 		  && 850	    && 800-899	&& 2 		     \\
		\vdots	  && \vdots  	&& \vdots	&& \vdots 	     \\
		n-2 	  && 520	    && 500-599	&& k 		     \\
		n-1       && 630	    && 600-699	&& k 		     \\
		n 		  && 610	    && 600-699	&& k 		     \\ \hline
	\end{tabular}
	\label{Tab: D-dataset-FSS}
\end{table}

An alternative way of label generation is to create bins with equal event counts per bin among the complete data set; \ac{fbs} labelling is shown in Table~\ref{Tab: D-dataset-FBS}.
\begin{table}[ht!]
	\centering
	\setlength{\arrayrulewidth}{0.75pt}%
	\caption{Data Set D with FBS Bin Size Labels.}
	\begin{tabular}{|cc|c|ccc|c|}
		\hline \rowcolor[HTML]{FFFFC7}
		\makecell{Event\\ID} 	&& Feature-A    	&& FBS Bins && \makecell{Sequence\\ID} \\ \hline
		1 	      && 280	    && 200-599	&& 1 		     \\
		2 		  && 250	    && 200-599	&& 1 		     \\
		3 	      && 890	    && 630-899	&& 2 		     \\
		4 		  && 850	    && 630-899	&& 2 		     \\
		\vdots	  && \vdots  	&& \vdots	&& \vdots 	     \\
		n-2 	  && 520	    && 200-599	&& k 		     \\
		n-1       && 630	    && 630-899	&& k 		     \\
		n 		  && 610	    && 600-629	&& k 		     \\ \hline
	\end{tabular}
	\label{Tab: D-dataset-FBS}
\end{table}
The alternative binning generation methods mentioned above let us derive two distinguished approaches to construct association networks. The first one is the \acs{fss} Network approach; it has graph nodes as binning groups with equal bin sizes. Manipulation of binning size allows us to aggregate events in different network nodes. The \acs{fbs} Network approach is the second one where the network nodes are binning groups with an equal number of events per bin. Defining a typical bucket size for the network nodes results in arbitrary interval boundaries for each node, and it allows to control their population.
 \begin{figure}[!ht]
	\begin{center}
		\makebox[\textwidth]{
			\centering
			\includegraphics[width=0.8\linewidth]{../images/methodology-association-networks-hyp_networks.png}}
		\caption{Graph Results For Two Different Network Approaches.}
		\label{figure-hyp_graphs}
	\end{center}
\end{figure}

Constructing \acs{fss} and \acs{fbs} networks concerning technology-driven constraints and load-driven constraints, respectively, for the production events, underlie the developed hypothesis of this thesis work: Non-random features of the association networks derived from these two approaches.