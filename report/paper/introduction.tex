\chapter{Introduction}

\section{Background Information and Motivation}\label{background_information}

An alloy of iron and carbon: steel has notable durability with superior mechanical characteristics. Based on the evolving ability of its microstructures~\cite{bhadeshia2017steels, Tasan2015}, steel compositions can be obtained in a wide range, and they can be recycled without loss of property. Those make steel an excellent material to meet the ever-changing requirements of our contemporary society.

A modern steel manufacturing facility has a complex structure consisting of various production lines and can process a set of products successively. Fundamental steel manufacturing steps with respective production units are shown in Fig.~\ref{figure-steel-production-steps}~\cite{sinha-spinks_2015}. Raw materials are melted in the blast, electric arc or basic oxygen furnaces to obtain liquid iron. In the next step, liquid steel alloy is sent to the continuous casting line. It is poured into a mould cavity where it starts to cool. Right after, it is treated in a secondary cooling process with water sprays until it solidifies. A further production line involves a rolling process, which can be performed in two different modes: hot rolling and cold rolling. It allows obtaining the desired mechanical properties of steel, uniform thickness, a control on width dimension, and converting material to a flat and rectangular slab, a semi-finished steel product. The solid steel is heated in a reheat furnace before it's subjected to high pressures in the hot rolling unit. Before the cold rolling process, steel is treated with pickling to remove rust and impurities on the slab surface; it makes it easier to work on the material. The cold rolling unit improves surface finish and flatness, and it allows to modify metal work hardening. After the rolling process, slabs might be converted into compact coils featuring high lengths unless they will not be sent to further process units on the continuous production line, such as a galvanising line. It is the application of protective zinc coating on the steel surface to improve corrosion resistance.

The whole continuous production process is scheduled in a sequencing fashion, and production events are grouped into production batches (the so-called production sequences). Production sequences are arranged by the common properties of production orders and priority. Various constraints arise in technical, logistic, physical and chemical aspects~\cite{cowling2001design, OZGUR2021106606} due to different machines and production lines; having efficient sequence planning is necessary to cope.
\clearpage

 \begin{figure}[!ht]
	\begin{center}
		\makebox[\textwidth]{
			\centering
			\includegraphics[width=1.0\linewidth]{../images/steel-production-steps.png}}
		\caption{Steel Production Steps~\cite{sinha-spinks_2015}.}
		\label{figure-steel-production-steps}
	\end{center}
\end{figure}

The above-explained processes and workstations can be arranged as a variety of integrated solutions based on the requirements of different demands or facility organisations. Table~\ref{Tab: production_lines} shows four production lines that the SMS group supplies~\cite{BRS}.
\begin{table}[ht!]
	\centering
	\setlength{\arrayrulewidth}{0.75pt}% 
	\caption{Compact Production Lines with Various Machine Modules}
	\begin{tabular}{|c|c|c|}
		\hline
		\cellcolor[HTML]{FFFFC7} 
		No & \cellcolor[HTML]{FFFFC7} \begin{tabular}[c]{@{}c@{}}Production \\ Unit / Line\end{tabular} & \cellcolor[HTML]{FFFFC7} Description \\ \hline
		1               & \makecell{Continuous Casting\\Machine (CCM)}            & \makecell{Steel cools, passing through \\ the mould cavity and \\solidifies after water spraying.}  \\ \hline
		2               & \makecell{Compact Strip \\Production (CSP)}              &\makecell{Compact plant including CCM,\\   reheating furnace, hot rolling unit \\and strip processing unit.}       \\ \hline
		3               & \makecell{Pickling Line \& Tandem \\Cold Mill (PLTCM)} & \makecell{Compact plant including a\\   turbulence pickling section \\and a tandem mill.}                     \\ \hline
		4               & \makecell{Continuous Galvanizing \\Line   (CGL)}         & \makecell{Application of protective zinc \\  coating on the steel surface to \\improve corrosion resistance.} \\ \hline
	\end{tabular}
	\label{Tab: production_lines}
\end{table}
 
Steel manufacturing factory systems are based on \ac{cim} implemented in an automated fashion with combined computer control and digital information~\cite{waldner1992}. The sensory information received from production line machines concerning production orders (the so-called production events) is stored as data within a server to be incorporated with planning progress to provide functionality, adaptability and effective resource allocation in manufacturing processes~\cite{Saadaoui2019}. A data collection of production events from a steel manufacturer was pulled from the SMS group database to investigate. The queries in \ac{sql} were generated to find the production events across $2$--$3$ years of production work completed in the production lines, introduced in Table~\ref{Tab: production_lines}. The SQL queries are attached as supplementary materials, \ref{figure-supplements-CCM_CSP-SQL}, \ref{figure-supplements-PLTCM-SQL}, and \ref{figure-supplements-CGL-SQL}. The query attributes and resulting attribute values are anonymous. Further details for the data collection and cleaning steps are given in Subsection~\ref{data_collection_cleaning}.

After manipulation and cleaning steps were performed, CCM, CSP, PLTCM, and CGL data sets have the number of events, $347418$, $205496$, $64026$, and $31230$, respectively. The decreasing number of events through the data sets shows that the output of a production line is not always an input for the next in line and might be excluded from downstream production lines, as previously mentioned. Moreover, the production events in PLTCM and CGL have a wide variety and precision in a narrow range for the thickness feature. In contrast, the events in CCM and CSP have a precise unimodal distribution gathered around single values in a broad range of thickness feature values as shown in Fig.~\ref{figure-thickness_diversity}.

 \begin{figure}[!ht]
	\begin{center}
		\makebox[\textwidth]{
			\centering
			\includegraphics[width=1.05\linewidth]{../images/introduction-thickness_diversity.png}}
		\caption{Diversity in Event Features for Different Production Lines.}
		\label{figure-thickness_diversity}
	\end{center}
\end{figure}

Production lines given in Table~\ref{Tab: production_lines} are listed in the decreasing order for the product portfolio range. Many semi-finished products might be excluded from further production steps to be delivered to consumers right after the casting and hot rolling units. As going further from CCM and CSP, the diversity of products gives its place to the more speciality among products in PLTCM and CGL.

 \begin{figure}[!ht]
	\begin{center}
		\makebox[\textwidth]{
			\centering
			\captionsetup{width=0.5\linewidth}
			\includegraphics[width=0.9\linewidth]{../images/introduction-hypothetical_constraints.png}}
		\caption{Different Categories of Production Events Handling. The hot rolling mill positioned in production line 1 treats only one slab at a time. The unit in production line 2, pickling tank full of acid coloured in grey, treats the metal surface of more than one slab at a time.
		}
		\label{figure-hypothetical_constraints}
	\end{center}
\end{figure}

In addition to product range differences, one can also see the production style differs between those production lines. Fig.~\ref{figure-hypothetical_constraints} illustrates alternative handling categories for the production events in two separate production lines. Production events in the sequences need to be arranged in a specific order according to their width dimension~\cite{OZGUR2021106606}, which is a technical constraint belonging to the unit in production line 1. In contrast, there is no such technical constraint for the surface treatment; yet, a certain level of production events volume can be treated, bringing a load constraint on the unit in production line 2.

\section{Research Objective and Plan}

We argue that technology-driven constraints arise from the handling type in production line 1 introduced in Fig.~\ref{figure-hypothetical_constraints}. In contrast, the other kind of handling in production line 2 in the exact figure causes load-driven constraints. Moreover, the technology-driven constraints are effective in CCM and CSP, while the load-driven constraints are effective in PLTCM and CGL, in our opinion.

Outgrowing questioning leads to asking if those two fundamentally different constraints can be correctly discriminated with a formal definition, leading us to observe functional consequences related to that. Hypothesising alternative binning schemes for those constraints is a way of quantifying them and would allow us to investigate if they impact how the production system behaves.

As the first step, we constructed an analysis pipeline on steel production data concerning the identified binning schemes, which would provide us with the statistical impacts to discriminate the two types of constraints. We used Big River Steel production data belonging to the production lines introduced in Table~\ref{Tab: production_lines} among $2$--$3$ years and analysed it in time intervals to observe the statistical impacts. As the second step, we developed an abstract theoretical framework. We ran experimental simulations that would help understand the difference between those constraints mechanistically and their statistical patterns.

The two steps mentioned above constitutes our \ac{or} model, which combines steel production events analysis and constraint-based simulation. The art form of this model is to structure a standard data format and a shared analysis logic that allows comparing the results from steel production data and simulation data.